%-------------------------
% Resume in Latex
% Author : Jake Gutierrez
% Based off of: https://github.com/sb2nov/resume
% License : MIT
%------------------------

\documentclass[letterpaper,11pt]{article}

\usepackage{latexsym}
\usepackage[empty]{fullpage}
\usepackage{titlesec}
\usepackage{marvosym}
\usepackage[usenames,dvipsnames]{color}
\usepackage{verbatim}
\usepackage{enumitem}
\usepackage[hidelinks]{hyperref}
\usepackage{fancyhdr}
\usepackage[english,russian]{babel} % added russian
\usepackage{tabularx}
\usepackage{calc} % added
\usepackage{ragged2e}

\input{glyphtounicode}


%----------FONT OPTIONS----------
% sans-serif
% \usepackage[sfdefault]{FiraSans}
% \usepackage[sfdefault]{roboto}
% \usepackage[sfdefault]{noto-sans}
% \usepackage[default]{sourcesanspro}

% serif
% \usepackage{CormorantGaramond}
% \usepackage{charter}


\pagestyle{fancy}
\fancyhf{} % clear all header and footer fields
\fancyfoot{}
\renewcommand{\headrulewidth}{0pt}
\renewcommand{\footrulewidth}{0pt}

% Adjust margins
\addtolength{\oddsidemargin}{-0.5in}
\addtolength{\evensidemargin}{-0.5in}
\addtolength{\textwidth}{1in}
\addtolength{\topmargin}{-.5in}
\addtolength{\textheight}{1.0in}

\urlstyle{same}

\raggedbottom
\raggedright
\setlength{\tabcolsep}{0in}

% Sections formatting
\titleformat{\section}{
  \vspace{-4pt}\scshape\raggedright\large
}{}{0em}{}[\color{black}\titlerule \vspace{-5pt}]

% Ensure that generate pdf is machine readable/ATS parsable
\pdfgentounicode=1

%-------------------------
% Custom commands
\newcommand{\resumeItem}[1]{
  \item\small{
    {#1 \vspace{-2pt}}
  }
}

\newcommand{\resumeSubheading}[4]{
  \vspace{-2pt}\item
    \begin{tabular*}{0.97\textwidth}[t]{l@{\extracolsep{\fill}}r}
      \textbf{#1} & #2 \\
      \textit{\small#3} & \textit{\small #4} \\
    \end{tabular*}\vspace{-7pt}
}

\newcommand{\resumeSubSubheading}[2]{
    \item
    \begin{tabular*}{0.97\textwidth}{l@{\extracolsep{\fill}}r}
      \textit{\small#1} & \textit{\small #2} \\
    \end{tabular*}\vspace{-7pt}
}

\newcommand{\resumeProjectHeading}[2]{
    \item
    \begin{tabular*}{0.97\textwidth}{l@{\extracolsep{\fill}}r}
      \small#1 & #2 \\
    \end{tabular*}\vspace{-7pt}
}

\newcommand{\resumeSubItem}[1]{\resumeItem{#1}\vspace{-4pt}}

\renewcommand\labelitemii{$\vcenter{\hbox{\tiny$\bullet$}}$}

\newcommand{\resumeSubHeadingListStart}{\begin{itemize}[leftmargin=0.15in, label={}]}
\newcommand{\resumeSubHeadingListEnd}{\end{itemize}}
\newcommand{\resumeItemListStart}{\begin{itemize}}
\newcommand{\resumeItemListEnd}{\end{itemize}\vspace{-5pt}}

%-------------------------------------------
%%%%%%  RESUME STARTS HERE  %%%%%%%%%%%%%%%%%%%%%%%%%%%%


\begin{document}

%----------HEADING----------
% \begin{tabular*}{\textwidth}{l@{\extracolsep{\fill}}r}
%   \textbf{\href{http://sourabhbajaj.com/}{\Large Sourabh Bajaj}} & Email : \href{mailto:sourabh@sourabhbajaj.com}{sourabh@sourabhbajaj.com}\\
%   \href{http://sourabhbajaj.com/}{http://www.sourabhbajaj.com} & Mobile : +1-123-456-7890 \\
% \end{tabular*}

\begin{center}
    \textbf{\Huge Александр Нельзин} \\ \vspace{2pt}
    \small +77073571374 $|$ \href{mailto:asnelzin@gmail.com}{\underline{asnelzin@gmail.com}} $|$
    \href{https://linkedin.com/in/asnelzin}{\underline{linkedin.com/in/asnelzin}} $|$
    \href{https://github.com/asnelzin}{\underline{github.com/asnelzin}}
\end{center}


%-----------EXPERIENCE-----------
\section{Опыт работы}
  \resumeSubHeadingListStart

    \resumeSubheading
      {Senior Software Engineer}{Ноябрь 2018 -- По настоящее время}
      {Togezzer.net, Inc.}{Удаленно, США}
      \resumeItemListStart
        \resumeItem{Разрабатываю backend для инновационного продукта "digital workspace": платформа для коммуникации и управления задачами/проектами/процессами компаний}
        \resumeItem{Занимаюсь, как продуктовыми, так и инфраструктурными задачами}
        \resumeItem{С нуля разработал и подключил систему для управления платежами и подписками клиентов}
      \resumeItemListEnd

% -----------Multiple Positions Heading-----------
%    \resumeSubSubheading
%     {Software Engineer I}{Oct 2014 - Sep 2016}
%     \resumeItemListStart
%        \resumeItem{Apache Beam}
%          {Apache Beam is a unified model for defining both batch and streaming data-parallel processing pipelines}
%     \resumeItemListEnd
%    \resumeSubHeadingListEnd
%-------------------------------------------

    \resumeSubheading
      {Разработчик}{Апрель 2018 -- Октябрь 2018}
      {Mail.ru Group}{Санкт-Петербург, Россия}
      \resumeItemListStart
        \resumeItem{Учавствовал в разработке и запуске более 5 проектов с самого начала, включая обсуждение бизнес-гипотез и быстрое прототипирование для проверки их жизнеспособности}
        \resumeItem{Подключал интеграции с другими проектами компании}
        \resumeItem{Проводил code-review, учил писать тесты, участвовал в планировании и ретроспективах}
    \resumeItemListEnd

    \resumeSubheading
      {Backend Developer}{Июль 2017 -- Апрель 2018}
      {ESforce Holding}{Санкт-Петербург, Россия}
      \resumeItemListStart
        \resumeItem{Участвовал в ранней стадии разработки микросервисного backend'а сервиса для видео-стриминга Looch.tv}
        \resumeItem{С нуля разработал сервис для записи видео-потоков в реальном времени}
        \resumeItem{Разрабатывал микросервисы с нуля, поддерживал имеющиеся: более 50к строк кода в git репозиторий}
        \resumeItem{Писал unit- и integration-тесты}
        \resumeItem{Профилировал программы и делал нагрузочные тесты}
      \resumeItemListEnd

    \resumeSubheading
      {Backend Developer}{Май 2016 -- Июнь 2017}
      {SputnikMobile}{Санкт-Петербург, Россия}
      \resumeItemListStart
        \resumeItem{Разработка веб-сервисов для американсого рынка: в частности, сервиса для выявления основных моментов (highlights) на видео-записях коммандных игр (футбол, баскетбол, регби)}
      \resumeItemListEnd

    \resumeSubheading
      {QA Engineer}{Ноябрь 2014 -- Май 2016}
      {Нетрика}{Санкт-Петербург, Россия}
      \resumeItemListStart
        \resumeItem{Ручное и автоматизированное тестирование}
        \resumeItem{Написание и поддержка инструментов для тестирования}
        \resumeItem{Настройка и поддержка CI/CD Jenkins}
      \resumeItemListEnd

    \resumeSubheading
      {Intern Python Developer}{Июль 2014 -- Август 2014}
      {Нетрика}{Санкт-Петербург, Россия}
      \resumeItemListStart
        \resumeItem{Разработка и тестирование нескольких проектов на Django в составе команды разработчиков}
      \resumeItemListEnd


  \resumeSubHeadingListEnd

%-----------EDUCATION-----------
\section{Высшее образование}
  \resumeSubHeadingListStart
    \resumeSubheading
      {Университет ИТМО}{Санкт-Петербург, Россия}
      {Бакалавр технических наук, специальность "Мехатроника и робототехника"}{Сентябрь 2011 -- Июль 2015}
  \resumeSubHeadingListEnd


%-----------PROGRAMMING SKILLS-----------
\section{Технические навыки}
 \begin{itemize}[leftmargin=0.15in, label={}]
    \small{\item{
     \textbf{Языки программирования}{: Go, Python, SQL} \\
     \textbf{Фреймворки}{: aiohttp, Django, FastAPI} \\
     \textbf{Базы данных}{: PostgreSQL, Redis, SQLite, BoltDB, MongoDB}\\
     \textbf{Архитектура}{: REST, gRPC, ZeroMQ, Microservices}\\
     \textbf{Распределенные системы}{: RabbitMQ, Apache Kafka}\\
     \textbf{Инструменты разработки}{: Git, Makefile, Prometheus, Grafana, Loki, nginx} \\
     \textbf{CI/CD, Cloud}{: GitLab CI/CD, Drone CI, Docker, Kubernetes, AWS}\\
    }}
 \end{itemize}


%-----------LANGUAGES-----------
\section{Знания языков}
 \begin{itemize}[leftmargin=0.15in, label={}]
    \small{\item{
     \textbf{Английский}{: Upper-intermediate (B2)} \\
     \textbf{Русский}{: Родной}
    }}
 \end{itemize}


%-------------------------------------------
\end{document}
